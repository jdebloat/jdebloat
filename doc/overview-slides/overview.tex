\documentclass[aspectratio=169]{beamer}

% Suppress all navigation symbols
\setbeamertemplate{navigation symbols}{}
\setbeamertemplate{headline}{}
\setbeamertemplate{footline}{}
\setbeamertemplate{itemize items}[circle]
\setbeamertemplate{footline}[frame number]

\setbeamercolor{structure}{fg=blue}
\setbeamercolor{normal text}{fg=black, bg=white}
\setbeamerfont{title}{series=\bfseries, size=\Huge}
\setbeamerfont{frametitle}{series=\bfseries, size=\LARGE}

\setbeamertemplate{footline}{
  \begin{tikzpicture}[remember picture,
                      overlay,
                      shift={(current page.south west)}]
    \node [black!50, inner sep=2mm, anchor=south east]
          at (current page.south east)
          {\large \bfseries \insertframenumber};
  \end{tikzpicture}
}

\definecolor{red}{RGB}{220, 50, 47}
\definecolor{green}{RGB}{133, 153, 0}
\definecolor{cyan}{RGB}{42, 161, 152}
\definecolor{blue}{RGB}{38, 139, 210}
\definecolor{yellow}{RGB}{181, 137, 0}

\usepackage{fontspec}
\setsansfont{Overpass}[Scale=MatchLowercase]
\setmonofont{Overpass Mono}[Scale=MatchLowercase]

\usepackage{listings}
\lstset{
  basicstyle=\ttfamily\small,
  commentstyle={\color{black!50}},
  language=Java,
}

\usepackage{pgfplots}

\usepackage{tikz}
\usetikzlibrary{arrows}
\usetikzlibrary{backgrounds}
\usetikzlibrary{calc}
\usetikzlibrary{decorations.pathreplacing}
\usetikzlibrary{fit}
\usetikzlibrary{positioning}
\usetikzlibrary{matrix}
\usetikzlibrary{scopes}
\usetikzlibrary{shapes}
\usetikzlibrary{tikzmark}
\usetikzmarklibrary{listings}

\title{JDebloat}
\author{Jon Eyolfson}
\date{2020-02-26}

\setbeamertemplate{title page}
{
  \begin{tikzpicture}[remember picture,
                      overlay,
                      shift={(current page.south west)}]
    \node (title) [inner sep=0, scale=1.2, align=center]
          at (\paperwidth / 2, \paperheight * 2 / 3)
          {\usebeamerfont{title}\usebeamercolor[fg]{title} \inserttitle};
    \node (author) [scale=1.5] at (\paperwidth / 2, \paperheight / 3)
          {\insertauthor};
    \node [anchor=south east, inner sep=2mm] at (\paperwidth, 0)
          {\bfseries \insertdate};
  \end{tikzpicture}
}

\begin{document}

  \begin{frame}[plain]
    \titlepage
  \end{frame}

  \setcounter{framenumber}{0}

  \begin{frame}
    \frametitle{Overall Benchmark Results}
    Before(7 Benchmarks)
    \begin{center}
      \begin{tabular}{c|r|r}
        \textbf{Tool} &
        \textbf{Mean Size Reduction(\%)} &
        \textbf{Median Size Reduction (\%)}\\
        \hline
        jinline & 20.0 & 19.8\\
        jshrink & 22.9 & 28.9\\
        jreduce & 15.4 & 0.6\\
        jinline+jshrink+jreduce & \textbf{38.5} & \textbf{31.0}\\
      \end{tabular}
    \end{center}

    \vspace{1em}

    After(25 Benchmarks)
    \begin{center}
      \begin{tabular}{c|r|r}
        \textbf{Tool} &
        \textbf{Mean Size Reduction(\%)} &
        \textbf{Median Size Reduction (\%)}\\
        \hline
        jinline & 14.6 & 18.1\\
        jshrink & 11.4 & 3.1\\
        jreduce & 40.24 & 30.4\\
        jinline+jshrink+jreduce & \textbf{49.9} & \textbf{30.3}\\
      \end{tabular}
    \end{center}
  \end{frame}

  \begin{frame}
    \frametitle{Push-Button Debloating}
    just \texttt{./jdebloat.py run}
  \end{frame}

  \begin{frame}[fragile]
    \frametitle{JInline Overview}
    \begin{columns}[t]
      \begin{column}{0.5\textwidth}
        \hspace{3.4em} \structure{Before}
        \begin{lstlisting}[xleftmargin=4em]
public class Application {
  // ...
  public void bar() {
    int x = a.foo();
  }
  // ...
}

public class LibraryA {
  // ...
  public int foo() {
    return /* complex */;
  }
  // ...
}
        \end{lstlisting}
      \end{column}
      \begin{column}{0.5\textwidth}
        \hspace{3.4em} \structure{After}
        \begin{lstlisting}[xleftmargin=4em]
public class Application {
  // ...
  public void bar() {
    int x = /* complex */;
  }
  // ...
}
        \end{lstlisting}
      \end{column}
    \end{columns}
    \begin{tikzpicture}[remember picture,
                        overlay,
                        shift={(current page.south west)}]
      \draw [blue, ->, >=stealth']
        ($(current page.center)+(-4em,-6.8em)$)
        -- node [above left, align=center, blue]
        {\footnotesize Inline \texttt{foo}}
        ($(current page.center)+(4em,0)$);
      \draw [blue, ->, >=stealth']
        ($(current page.center)+(-4em,-6.8em)$)
        -- node [below, align=center, blue]
        {\footnotesize Remove \texttt{LibraryA}}
        ($(current page.center)+(4em,-6.8em)$);
    \end{tikzpicture}
  \end{frame}

  \begin{frame}
    \frametitle{JInline Delta}
    Enable inlining of multiple targets

    More aggressive inlining might reduce immediate reduction, but helps later
  \end{frame}

  \begin{frame}
    \frametitle{JInline}
    Gerenuk, published in SOSP, ``object debloating''
  \end{frame}

  \begin{frame}
    \frametitle{JInline Future Work}
    Currently inlines all targets that we can

    To increase synergy, only inline at a callsite if it will help the other
    tools
  \end{frame}

  \begin{frame}
    \frametitle{JShrink Overview}
    JShrink works by generating a static call graph of an input program.
    It proceeds to remove methods that are not used based on static call graph
    analysis.
    When using JShrink, the user is required to specify entry points for
    constructing the call graph.
    JShrink provides three pre-programmed options: (1) all main methods, (2) all
    public methods (excluding tests), and/or (3) all JUnit Tests. The user may
    also specify custom entry points if required.
  \end{frame}

  \begin{frame}
    \frametitle{JShrink Delta}
    JMTrace(java agent) to handle dynamic class loading
  \end{frame}

  \begin{frame}
    \frametitle{JShrink Research Output}

    JShrink under submission
  \end{frame}

  \begin{frame}
    \frametitle{JShrink Future Work}
        ??
  \end{frame}

  \begin{frame}
    \frametitle{JReduce Overview}
    JReduce is a tool that uses a variant of delta-debugging to reduce the
    classes of a project given a property.
  \end{frame}

  \begin{frame}
    \frametitle{JReduce Delta}

    Last year: focus on reducing number of classes

    Delta this year: extend idea to also remove methods and fields
  \end{frame}

  \begin{frame}
    \frametitle{JReduce Research Output}

    JReduce was published in FSE'19
  \end{frame}

  \begin{frame}
    \frametitle{JReduce Future Work}

    Our current work, which I'm also going to present. Is improving the
    modeling language from Graphs to Logics. This enable us to get around 8
    times better reduction than [FSE'19], while only being 2-3 times slower
    (preliminary numbers).
  \end{frame}
\end{document}
